\documentclass[12pt]{article}
 
\usepackage[margin=1in]{geometry} 
\usepackage{amsmath,amsthm,amssymb,enumitem, wasysym}
\usepackage{slashed}
\usepackage{tikz}
\newcommand{\N}{\mathbb{N}}
\newcommand{\Z}{\mathbb{Z}}
\DeclareMathOperator{\Q}{\mathbb{Q}}
\DeclareMathOperator{\R}{\mathbb{R}}
\usepackage[T1]{fontenc}
 
\newenvironment{lemmaNoNum}[2][Lemma 1:]{\begin{trivlist}
\item[\hskip \labelsep {\bfseries #1}]}{\end{trivlist}}

\begin{document}

\textbf{Calculation of BOLA parameters $V$ and $\gamma$}  
\begin{itemize}
\setlength\itemsep{0.3em}
    \item 
$V$, $\gamma$, and $p$: see paper 
\item
$S_{-1}$: size of smallest chunk 
\item
$u_{-1}$: utility of smallest chunk 
\item
$S_{-2}$: size of second-smallest chunk (TODO: handle $S_{-2} == S_{-1}$)
\item
$u_{-2}$: utility of second-smallest chunk 
\item
$u_{0}$: highest utility 
\item
$MAXBUF$: Buffer level (chunks) at which largest chunk has an objective value of zero
\item
$LOWBUF$: Buffer level (chunks) at which smallest chunk has an objective value of zero (\textbf{not} buffer level below which smallest chunk is always chosen)
% TODO: check equation numbering 
\end{itemize}
(Note: size units don't impact parameter calculation, but utility units do.)

 \begin{align}
\intertext{First, directly from the solution to (9) in the paper$^1$:}
V(u_{0} + \gamma p) &= MAXBUF 
\intertext{Solve (1) for $V$: }
V &=\frac{MAXBUF}{u_{0} + \gamma p} 
\intertext{By definition of $LOWBUF$, the smallest format has an objective value of zero when}
     Q &= LOWBUF 
    \intertext{Substitute (3) into the definition of objective: }
    \frac{V(u_{-1} + \gamma p) - LOWBUF}{S_0} &= 0
\intertext{Solve (4) for $V$:}
V &= \frac{LOWBUF}{u_{-1} + \gamma p} 
\intertext{Equate the two expressions for $V$ from (2) and (5):} 
\frac{MAXBUF}{u_{0} + \gamma p} &= \frac{LOWBUF}{u_{-1} + \gamma p} 
\intertext{Solve (6) for $\gamma p$:} 
\gamma p &= \frac{LOWBUF(u_0) - MAXBUF(u_{-1})}{MAXBUF - LOWBUF}
\intertext{Substitute (7) into (2) to solve for $V$.}\nonumber
\end{align}

\newpage
\textbf{Sign of $\gamma$} \\\\
The paper requires $\gamma > 0$, presumably since the goal is to maximize $\overline{v} + \gamma \overline{s}$, where $\overline{v}$ is average utility and $\overline{s}$ is average fraction of time spent \textbf{not} rebuffering.\\\\
Sign is easier to reason about if $u_{-1} = 0$, but paper does not require this. \\\\
\textbf{If} we take $u_{-1} = 0$ in (7), then we have 
\setcounter{equation}{0}
\begin{align} 
\gamma p = \frac{LOWBUF(u_0)}{MAXBUF - LOWBUF} &> 0
\intertext{Note that the denominator in (1) is guaranteed to be \textbf{positive}, giving}
LOWBUF(u_0) &> 0
\intertext{If $u_0 > 0$ (paper does not explicitly require this), then (2) is true, guaranteeing $\gamma > 0$.}
\intertext{\textbf{However}, if we do not assume the value of $u_{-1}$, again noting that the denominator in (7) is positive, then $\gamma > 0$ only implies that}
LOWBUF(u_0) - MAXBUF(u_{-1}) &> 0
\end{align}

\newpage
\textbf{Dash.js parameter calculation} \\\\
Let $DASHBUF = 10 / p$ (from constant $MINIMUM\_BUFFER\_S$ = 10 in code). \\ 

From a comment (consistent with their expressions for $V$ and $\gamma p)$: 
\setcounter{equation}{0}
\begin{align}
V(u_{-1} + \gamma p) &= DASHBUF
\intertext{Rearrange:}
V(u_{-1} + \gamma p) - DASHBUF &= 0
\intertext{Notice (2) implies that the objective for the smallest chunk has value 0 at DASHBUF.}
\end{align}

\newpage
\textbf{Footnotes} \\
\begin{enumerate}
    \item 
    Derivation (I don't see this in the paper): \\
    
    Let $S_0$ be the size of the largest format.
    The largest format has an objective value of zero when 
    \setcounter{equation}{0}
    \begin{align} 
    Q &= MAXBUF 
    \intertext{Substitute (1) into the definition of objective: }
    \frac{V(u_0 + \gamma p) - MAXBUF}{S_0} &= 0
    \intertext{Solve (2) for MAXBUF:}
    V(u_{0} + \gamma p) &= MAXBUF 
    \end{align}
  
\end{enumerate}

\end{document}

