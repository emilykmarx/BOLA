\documentclass[12pt]{article}
 
\usepackage[margin=1in]{geometry} 
\usepackage{amsmath,amsthm,amssymb,enumitem, wasysym}
\usepackage{slashed}
\usepackage{tikz}
\newcommand{\N}{\mathbb{N}}
\newcommand{\Z}{\mathbb{Z}}
\DeclareMathOperator{\Q}{\mathbb{Q}}
\DeclareMathOperator{\R}{\mathbb{R}}
\usepackage[T1]{fontenc}
 
\newenvironment{lemmaNoNum}[2][Lemma 1:]{\begin{trivlist}
\item[\hskip \labelsep {\bfseries #1}]}{\end{trivlist}}

\begin{document}

\textbf{Calculation of BOLA parameters $V$ and $\gamma$}  
\begin{itemize}
\setlength\itemsep{0.3em}
    \item 
$V$, $\gamma$, and $p$: see paper 
\item
$S_{-1}$: size of smallest chunk 
\item
$u_{-1}$: utility of smallest chunk 
\item
$S_{-2}$: size of second-smallest chunk (TODO: handle $S_{-2} == S_{-1}$)
\item
$u_{-2}$: utility of second-smallest chunk 
\item
$u_{0}$: highest utility 
\item
$MAXBUF$: Buffer level (chunks) above which no chunk is sent (only considering buffer's contribution to the decision - Puffer media server has more criteria) 
\item
$MINBUF$: Buffer level (chunks) at which objectives are equal for two smallest formats. Assume $MAXBUF > MINBUF$.
\end{itemize}
(Note: size units don't impact parameter calculation, but utility units do.)
\begin{align}
\intertext{First, directly from the solution to (9) in the paper:}
V(u_{0} + \gamma p) &= MAXBUF 
\intertext{Solve (1) for $V$: }
V &=\frac{MAXBUF}{u_{0} + \gamma p} 
\intertext{Second, from intersecting the objectives of the smallest and second-smallest chunks (an extrapolation from the paper):}
\frac{V(u_{-1} + \gamma p) - MINBUF}{S_{-1}} &= \frac{V(u_{-2} + \gamma p) - MINBUF}{S_{-2}} 
\intertext{Solve (3) for $V$:}
V &= \frac{MINBUF(S_{-2} - S_{-1})}{S_{-2}(u_{-1} + \gamma p) - S_{-1}(u_{-2} + \gamma p)}  
\intertext{Equate the two expressions for $V$ from (2) and (4):} 
\frac{MAXBUF}{u_{0} + \gamma p} &= \frac{MINBUF(S_{-2} - S_{-1})}{S_{-2}(u_{-1} + \gamma p) - S_{-1}(u_{-2} + \gamma p)} 
\intertext{Solve (5) for $\gamma p$:} 
\gamma p &= \frac{MAXBUF(S_{-2}u_{-1} - S_{-1}u_{-2}) - u_{0}MINBUF(S_{-2}-S_{-1})}{(MINBUF - MAXBUF)(S_{-2} - S_{-1})} 
\intertext{Substitute (6) into (2) to solve for $V$.}\nonumber
\end{align}

\newpage
\textbf{Sign of $\gamma$} \\\\
The paper requires $\gamma > 0$, presumably since the goal is to maximize $\overline{v} + \gamma \overline{s}$, where $\overline{v}$ is average utility and $\overline{s}$ is average fraction of time spent \textbf{not} rebuffering.\\\\
Sign is easier to reason about if $u_{-1} = 0$, but paper does not require this. \\\\
\textbf{If} we take $u_{-1} = 0$ in (6), then we have 
\begin{align}
\gamma p = \frac{-MAXBUF(S_{-1}u_{-2}) - u_{0}MINBUF(S_{-2}-S_{-1})}{(MINBUF - MAXBUF)(S_{-2} - S_{-1})} &> 0
\intertext{Note that the denominator in (7) is guaranteed to be \textbf{negative}, giving}
-MAXBUF(S_{-1}u_{-2}) - u_{0}MINBUF(S_{-2}-S_{-1}) &< 0
\intertext{Solve (8) for an expression wrt utility, assuming $u_{-2} > 0$:}
\frac{u_0}{u_{-2}} &> \frac{-MAXBUF(S_{-1})}{MINBUF(S_{-2} - S_{-1})}
\intertext{If utilities are nonnegative (paper does not explicitly require this), then LHS of (9) is nonnegative and RHS is negative, guaranteeing $\gamma > 0$.}
\intertext{\textbf{However}, if we do not assume the value of $u_{-1}$, again noting that the denominator in (6) is negative, then $\gamma > 0$ only implies that}
MAXBUF(S_{-2}u_{-1} - S_{-1}u_{-2}) - u_{0}MINBUF(S_{-2}-S_{-1}) &< 0
\end{align}
\end{document}

